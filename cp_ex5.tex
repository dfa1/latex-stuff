\documentclass{article}
\RequirePackage{amssymb}
\RequirePackage{amsmath}
\usepackage[latin1]{inputenc}

\title{Esercizio 5 di riepilogo}
\author{Davide Angelocola}

\begin{document}
\maketitle

\section{Testo del problema}
Un'apparecchiatura di et� $j \ge 0$ all'inizio della giornata si guasta
durante la giornata con probabilit� $p_j$ e in tal caso � sostituita
da un'apparecchiatura identica ma nuova, che entra in funzione
all'inizio della giornata successiva. L'apparecchiatura � sostituita anche quando � troppo vecchia e si conviene che questo corrisponda all'et� N (per et� di un'apparecchiatura si intende qui
il numero delle giornate intere in cui l'apparecchiatura ha funzionato). Per $n \ge 0$, si indica con $X_n$ la v.a. che conta l'et�
dell'apparecchiatura funzionante all'inizio della $n$+1-ma giornata.

\begin{enumerate}
    \item Scrivi la matrice di transizione della catena di Markov $(X_n)_{n \ge 0}$ e determina il carattere degli stati.
 \item Se al tempo $n = 0$ l'apparecchiatura � nuova, con che probabilit� � nuova al tempo $n = 2$?

\end{enumerate}

\section{Svolgimento}


\end{document}

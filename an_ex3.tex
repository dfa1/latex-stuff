\documentclass{article}
\RequirePackage{amssymb}
\RequirePackage{amsmath}
\usepackage[latin1]{inputenc}

\title{Esercizio 3}
\author{Davide Angelocola}

\begin{document}
\maketitle

\section{Problema}

Sia 

$$ A = \begin{pmatrix} 
  2 & 1 & 1 \cr
  1 & 2 & p \cr
  1 & p & 2 \cr
\end{pmatrix}$$

Far vedere per quali $p$ si impiegano due passi per convergere.

\section{Svolgimento}

Si fa uso di una matrice \textit{Givens} $3x3$ definita come:

$$S = \begin{pmatrix}
 \alpha & \beta & 0  \cr 
 -\beta & \alpha & 0 \cr  
 0 & 0 & 1           \cr
\end{pmatrix}$$

Quindi si procede con i prodotti:

$$A = S^{-1}AS$$

partendo da $S^{-1}A$:

$$\left(
\begin{array}{ccc}
 \alpha  & -\beta  & 0 \\
 \beta  & \alpha  & 0 \\
 0 & 0 & 1
\end{array}
\right)\left(
\begin{array}{ccc}
 2 & 1 & 1 \\
 1 & 2 & p \\
 1 & p & 2
\end{array}
\right) = \begin{pmatrix}2\alpha -\beta & \alpha -2 \beta & \alpha -p \beta \cr, \alpha +2 \beta & 2 \alpha +\beta & p \alpha +\beta \cr 1 & p \end{pmatrix}\begin{pmatrix}
 \alpha & \beta & 0  \cr 
 -\beta & \alpha & 0 \cr  
 0 & 0 & 1           \cr
\end{pmatrix}$$

per poi calcolare il prodotto per $S$, ottenendo la matrice:

$$\left(
\begin{array}{ccc}
 \alpha  (2 \alpha -\beta )-(\alpha -2 \beta ) \beta  & \alpha  (\alpha -2 \beta )+(2 \alpha -\beta ) \beta  & \alpha -p \beta  \\
 -\beta  (2 \alpha +\beta )+\alpha  (\alpha +2 \beta ) & \alpha  (2 \alpha +\beta )+\beta  (\alpha +2 \beta ) & p \alpha +\beta  \\
 \alpha -p \beta  & p \alpha +\beta  & 2
\end{array}
\right)$$

Per definizione $\alpha - \beta p$, si annulla al primo passo. Per ottenere una matrice diagonale in due passi occorre porre 

$$\alpha p + \beta = 0$$ 

da cui si ottiene:

$$p = -\frac{\beta}{\alpha} $$

\end{document}

\documentclass{article}
\RequirePackage{amssymb}
\RequirePackage{amsmath}
\usepackage[latin1]{inputenc}

\title{Esercizio 3}
\author{Davide Angelocola}

\begin{document}
\maketitle

\section{Problema}

Sia 

$$ A = \begin{pmatrix} 
  2 & 1 & 1 \cr
  1 & 2 & p \cr
  1 & p & 2 \cr
\end{pmatrix}$$

Far vedere per quali $p$ si impiegano due passi per convergere.

\section{Svolgimento}

Esistono diversi criteri per scegliere quale elemento azzerare, tra di essi ne consideriamo due, per i quali si pu� esibire la dimostrazione di convergenza:

\begin{itemize}
    \item al generico passo si sceglie l'elemento pi� grande in modulo
    \item al generico passo si sceglie l'elemento con indici $i < j$
\end{itemize}



\subsection{Criterio del pi� grande in modulo}

La matrice $A$ ha gli elementi sopra la diagonale pari a 1, quindi supponendo di adottare il primo criterio di scelta si presenterebbero 3 casi: 
$$ p = 
\begin{cases}
   0 & in\ questo\ caso\ il\ metodo\ di\ Jacobi\ converge\ in\ un\ passo\\
   1 & che\ abbiamo\ visto\ convergere\ in\ due\ passi\ in\ un\ altro\ esercizio \\
   altrimenti &  si\ deve\ cominciare\ ad\ annullare\ proprio\ da\ questo\ elemento
\end{cases}
$$

L'unico caso interessante risulta quindi essere quello in cui $p > 1$. Si applica il metodo di Jacobi, annullando proprio l'elemento posto in $A_{2 3}$ usando la seguente matrice di \textit{Givens} $3x3$ definita come:

$$S = \begin{pmatrix}
 1 & 0 & 0           \cr
 0 & \alpha & \beta  \cr 
 0 & -\beta & \alpha \cr  
\end{pmatrix}$$

Applicando la prima trasformazione:

$$A = S^{-1}AS$$

$$
=\left(
\begin{array}{ccc}
 1 & 0 & 0 \\
 0 & \alpha  & -\beta  \\
 0 & \beta  & \alpha 
\end{array}
\right).\left(
\begin{array}{ccc}
 2 & 1 & 1 \\
 1 & 2 & p \\
 1 & p & 2
\end{array}
\right).\left(
\begin{array}{ccc}
 1 & 0 & 0 \\
 0 & \alpha  & \beta  \\
 0 & -\beta  & \alpha 
\end{array}
\right)
=$$

$$=\left(
\begin{array}{ccc}
 2 & \alpha -\beta  & \alpha +\beta  \\
 \alpha -\beta  & 2 \alpha ^2-2 p \alpha  \beta +2 \beta ^2 & p \alpha ^2-p \beta ^2 \\
 \alpha +\beta  & p \alpha ^2-p \beta ^2 & 2 \alpha ^2+2 p \alpha  \beta +2 \beta ^2
\end{array}
\right)
$$

Risulta quindi evidente che $p\alpha^2 - p\beta^2$ si annulla se e solo se 
$\alpha^2 - \beta^2 = 0$. Dopo il primo passo  si ottiene la  matrice $A'$:

$$A' = 
\begin{pmatrix}
 2 & \alpha -\beta  & \alpha +\beta  \cr
 \alpha -\beta  & 2 \alpha ^2-2 p \alpha  \beta +2 \beta ^2 & 0\cr
 \alpha +\beta  & 0 & 2 \alpha ^2+2 p \alpha  \beta +2 \beta ^2 \cr
\end{pmatrix}
$$ 

\textbf{NB}: se riapplico una trasformazione ottengo una matrice con termini troppo complicati (equazioni di terzo grado in $\alpha$ e $\beta$). 

\subsection{Critero sequenziale}

Al generico passo scelgo l'elemento da annullare come $A_{i j}$ con $i < j$. Nel caso della matrice di $A$ si comincia con l'annullare elemento posto in $A_{1 2}$ usando la matrice di \textit{Givens} $3x3$ cos� definita:

$$S = \begin{pmatrix}
 \alpha & \beta & 0  \cr 
 -\beta & \alpha & 0 \cr  
 0 & 0 & 1           \cr
\end{pmatrix}$$

Si applica la trasformazione:

$$A = S^{-1}AS$$

partendo da $S^{-1}A$:

$$\left(
\begin{array}{ccc}
 \alpha  & -\beta  & 0 \\
 \beta  & \alpha  & 0 \\
 0 & 0 & 1
\end{array}
\right)\left(
\begin{array}{ccc}
 2 & 1 & 1 \\
 1 & 2 & p \\
 1 & p & 2
\end{array}
\right) = \begin{pmatrix}2\alpha -\beta & \alpha -2 \beta & \alpha -p \beta \cr \alpha +2 \beta & 2 \alpha +\beta & p \alpha +\beta \cr 1 & p \end{pmatrix}\begin{pmatrix}
 \alpha & \beta & 0  \cr 
 -\beta & \alpha & 0 \cr  
 0 & 0 & 1           \cr
\end{pmatrix}$$

per poi calcolare il prodotto per $S$, ottenendo la matrice:

$$=\left(
\begin{array}{ccc}
 \alpha  (2 \alpha -\beta )-(\alpha -2 \beta ) \beta  & \alpha  (\alpha -2 \beta )+(2 \alpha -\beta ) \beta  & \alpha -p \beta  \\
 -\beta  (2 \alpha +\beta )+\alpha  (\alpha +2 \beta ) & \alpha  (2 \alpha +\beta )+\beta  (\alpha +2 \beta ) & p \alpha +\beta  \\
 \alpha -p \beta  & p \alpha +\beta  & 2
\end{array}
\right)=$$

$$
=\left(
\begin{array}{ccc}
 2 \left(\alpha ^2-\alpha  \beta +\beta ^2\right) & \alpha ^2-\beta ^2 & \alpha -p \beta  \\
 \alpha ^2-\beta ^2 & 2 \left(\alpha ^2+\alpha  \beta +\beta ^2\right) & p \alpha +\beta  \\
 \alpha -p \beta  & p \alpha +\beta  & 2
\end{array}
\right)
$$

Si devono assegnare due valori ad $\alpha$ e $\beta$ in modo tale da annullare l'elemento $A_{1 2}$ (e $A_{2 1}$). All'inizio del secondo passo si ha quindi la seguente matrice:

$$
A'=\left(
\begin{array}{ccc}
 2 \left(\alpha ^2-\alpha  \beta +\beta ^2\right) & 0 & \alpha -p \beta  \\
 0 & 2 \left(\alpha ^2+\alpha  \beta +\beta ^2\right) & p \alpha +\beta  \\
 \alpha -p \beta  & p \alpha +\beta  & 2
\end{array}
\right)
$$

Si impone ora che gli elementi posti in $A_{1,3}$ e $A_{2, 3}$ si abbiano due elementi nulli:

$$\alpha - p\beta = 0$$
$$p\alpha + \beta = 0$$

e da ci� si ottiene:

$$\alpha - p\beta = p\alpha + \beta$$
$$p(\alpha+\beta) = \alpha - \beta$$
$$p = \frac{\alpha - \beta}{\alpha + \beta}$$
 





\end{document}

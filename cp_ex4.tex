\documentclass{article}
\RequirePackage{amssymb}
\RequirePackage{amsmath}
\usepackage[latin1]{inputenc}

\title{Esercizio 4 di riepilogo}
\author{Davide Angelocola}

\begin{document}
\maketitle

\section{Testo del problema}

Un uomo possiede 2 ombrielli e ne prende uno al mattino per andare in ufficio
e a sera quando torna, naturalmente se piove e se ce n'� uno disponibile. Assumi che ogni volta la probabilit� che piova sia $p$. Indica con $(X_n)_{n\ge 1})$ la catena di Markov che conta gli ombrelli disponibili prima dell'$n$-mo tragitto (senza distinguere tra tragitti di andata di di ritorno).

\begin{enumerate}
  \item Scrivi la matrice di transizione della catena e determina il carattere degli stati.
  \item Con che probabilit� l'uomo non ha ombrelli disponibili prima del secondo, del terzo e del quarto tragitto se si assume che prima del primo tragitto li abbia entrambi disponibili (la legge iniziale di $X_1$ � la delta di Dirac in 2)?
  \item Dopo quanto tempo in media, se si assume che prima del primo tragitto abbia entrambi gli ombrelli disponibili, l'uomo non ha ombrelli disponibili?
\end{enumerate}

\section{Svolgimento}

\end{document}

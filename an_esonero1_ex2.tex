\documentclass{article}
\RequirePackage{amssymb}
\RequirePackage{amsmath}
\usepackage[latin1]{inputenc}

\title{Esercizio 2 - Esonero 1}
\author{Davide Angelocola}

\begin{document}
\maketitle

\section{Problema}

Utilizzando $3$ valori $f(x) = \frac{1}{x+1}$ nell'intervallo $[1, \frac{3}{2}]$ fornire un'approssimazione di $\int_1^{\frac{3}{2}}{f(x)dx}$.

\section{Svolgimento}

Per prima cosa si pone $n = 3$, ottenendo quindi due trapezi:

$$h = \frac{(b - a)}{n} = \frac{\frac{3}{2} - 1}{2} = \frac{1}{4}$$ 

ora si pu� calcolare:

$$I_h\textdegree = h [ \frac{f(a)}{2}+f(a+h)+\frac{f(b)}{2}]=$$
$$=\frac{1}{4}[\frac{f(1)}{2} + f(1+\frac{1}{4}) + \frac{f(\frac{3}{2})}{2}]=$$
$$=\frac{1}{4}[\frac{1}{4}+\frac{4}{9}+\frac{1}{5}]=$$
$$=\frac{161}{720}=0.223611$$ 

Questo risultato � preciso alla terza decisa cifra decimale. Possiamo fare di meglio? 

Si prova ad applicare il metodo di \textit{Romberg}. Ponendo $k=1$ quindi dobbiamo calcolare: $$I^{j}_{\frac{h}{2^k}}=\frac{2^{2j}(I^{j-1}_{\frac{h}{2^k}}-I^{j-1}_\frac{h}{2^{k-1}})}{2^{2j} - 1}$$\label{1}

Ponendo quindi $h = 1/4 * (1/2^k) = 1/8$ si ottiene un'approssimazione in 5 punti (4 trapezi):

$$I_\frac{h}{2^{k}} = h [ \frac{f(a)}{2}+f(a+h)+f(a+2h)+f(a+3h)+\frac{f(b)}{2}]=$$
$$=\frac{1}{8}[\frac{f(1)}{2} + f(1+\frac{1}{8}) + f(1+\frac{2}{8}) + f(1+\frac{3}{8}) + \frac{f(\frac{3}{2})}{2}]=$$
$$=\frac{1}{8}[\frac{1}{4}+\frac{8}{17}+\frac{4}{9}+\frac{8}{19}+\frac{1}{5}]=$$
$$=\frac{103843}{465120}=0.223261$$

Si ottiene, con applicando $j=1$ nella \ref{1} la  infine:

$$I\textdegree = \frac{2^2I_{\frac{h}{2}} - I_h}{2^2-1}=$$
$$=\frac{2^2\frac{103843}{465120} - \frac{161}{720}}{3}=$$
$$=\frac{\frac{415372}{465120}-\frac{161}{720}}{3}=$$
$$=0.223144$$.

\section{Svolgimento al computer}

Sapendo che la primitiva di $\frac{1}{1+x}$ � $log(x+1)$:

$$\int_1^{\frac{3}{2}}{\frac{1}{1+x}dx} = log(\frac{3}{2}+1) - log(2) = log(\frac{5}{4}) = 0.2231\_4355\_1314\_2097$$

\end{document}

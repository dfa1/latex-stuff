\documentclass{article}
\RequirePackage{amssymb}
\RequirePackage{amsmath}
\usepackage[latin1]{inputenc}

\title{Esercizio 2 - Esonero 1}
\author{Davide Angelocola}

\begin{document}
\maketitle

\section{Problema}

Utilizzando 3 valori $f(x) = \frac{1}{x+1}$ nell'intervallo $[1, \frac{3}{2}]$ fornire un'approssimazione di $\int_1^{\frac{3}{2}}{f(x)dx}$.

\section{Svolgimento}

Per prima cosa si pone $n = 3$ e quindi $h$ vale:

$$h = \frac{(b - a)}{n} = \frac{\frac{3}{2} - 1}{3} = \frac{1}{6}$$ 

ora si pu� calcolare $$I_h\textdegree = h [ \frac{f(a)}{2}+f(a+h)+f(a+2h)+\frac{f(b)}{2}]=$$

$$=\frac{1}{6}[\frac{f(1)}{2} + f(1+\frac{1}{6}) + f(1+\frac{2}{6})+\frac{f(\frac{3}{2})}{2}] = $$

svolgendo i conti si ottiene:

$$ = \frac{813}{3640} \approx 0.2233\_5164$$

Applicando il metodo di Romberg si ottiene:


\section{Svolgimento al computer}

Sapendo che la primitiva di $\frac{1}{1+x}$ � $log(x+1)$:

$$\int_1^{\frac{3}{2}}{\frac{1}{1+x}dx} = log(\frac{3}{2}+1) - log(2) = log(\frac{5}{4}) = 0.2231\_4355\_1314\_2097$$

\end{document}

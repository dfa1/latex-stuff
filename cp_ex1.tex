\documentclass{article}
\RequirePackage{amssymb}
\RequirePackage{amsmath}
\usepackage[latin1]{inputenc}

\title{Esercizio 1 di riepologo}
\author{Davide Angelocola}

\begin{document}
\maketitle

\section{Problema}

Sono date due urne e $2d$ biglie di cui $d$ nere e $d$ bianche. Inizialmente $d$ biglie scelte a caso sono poste nell'urna 1 e le restanti $d$ sono collocate nell'urna 2. Ad ogni istante una biglia � scelta a caso da ciascuna delle due urne ed � spostata nell'altra urna. Sia $X_0$ il numero iniziale delle biglie nere nell'urna 1 e $X_n$ il numero delle biglie nere nell'urna 1 al tempo $n$. Trovare le probabilit� di transizione della catena di Markov $(X_n)_{n\ge0}$.

\section{Svolgimento}

Ad ogni estrazione di possono verificare tutte le combinazioni (4) degli eventi pesco una biglia dalla prima urna e dalla seconda. Gli eventi sono:

\begin{enumerate}
    \item pesco una biglia nera dalla prima urna e dalla seconda urna (NN)
    \item pesco una biglia nera della prima urna e una bianca dalla seconda (NB)
    \item pesco una biglia bianca dalla prima urna e dalla seconda (BB)
    \item pesco una biglia bianca dalla prima urna e una nera dalla seconda (BN)
\end{enumerate}

Sono tutti eventi indipendenti ma a noi interessano solo gli eventi che cambiano il numero di biglie nere nella prima urna. Si esclude quindi l'evento (BB), poich� non altera il numero di biglie nere. Escludiamo inoltre l'evento (NN) poich� anch'esso non altera il numero di biglie nere essendo un mero scambio di biglie. 

Studiamo quindi come si modifica la variabile $X_n$ in funzione dell'evento che si verifica:

$$ X_n = \begin{cases}
  (BN) & +1 \\ 
  (NB) & -1 \\
  (BB) & 0  \\
  (NN) & 0  \\
\end{cases}
$$

Notiamo subito che il numero delle biglie in ciascuna delle urne � sempre pari a $d$ a prescindere dalla sequenza di eventi che si verifica. Essendo gli eventi indipendenti si tratta di una catena di Markov. Si ha inoltre il sospetto che sia una catena di nascita e morte. Proviamo a dimostrare queste supposizioni.

L'evento (BN) tale che incrementa $X_n$ si verifica se vengono verificati gli eventi (B) nella prima urna e (N) nella seconda, ovvero che viene pescata una biglia bianca dalla prima urna \textbf{e} biglia nera dalla seconda. Sia $i$ il numero di biglie nere nella prima urna, essendo gli eventi indipendenti di probabilit� si ha che:

$$
P(B) = \frac{d-i}{d}
$$
$$
P(N) = \frac{d-i}{d}
$$
$$
P(BN) = P(B)*P(N) = \frac{d-i}{d} * \frac{d-i}{d} = \frac{(d-i)^2}{d}
$$

La probabilit� vengono calcolate banalmente come casi favorevoli su casi possibili. 

L'evento (NB) si verifica quando si pesca una biglia nera dalla prima urna e una bianca dalla seconda:

$$
P(N) = \frac{i}{d}  
$$
$$
P(B) = \frac{i}{d}  
$$
$$
P(NB) = \frac{i}{d} \frac{i}{d} = (\frac{i}{d})^2
$$

Gli eventi (NN) e (BB) come gi� asserito in precedenza non modificano $X_n$ e si verificano con probabilit� uguali (per ipotesi ci sono $d$ palline nere e $d$ palline bianche):

$$
P(N) = P(B) = \frac{i}{d}   
$$

per quanto riguarda la prima urna. Dato che il numero totale delle biglie nere (o bianche) deve essere $d$ per la seconda urna la probabilit� di estrarre una pallina dello stesso colore � pari a: 
$$
P(N) = P(B) = \frac{d-i}{d}  
$$
$$
P(NN) = P(BB) = (\frac{i}{d})(\frac{d-i}{d}) 
$$

Notiamo che si tratta di una catena di nascita e morte. Avendo ora scritto tutte le probabilit� possiamo determinare i parametri $p$, $q$ e $r$ della catena:

$$
p_i = P(X_{n+1} = i + 1 | X_n = i) = P(BN) = \frac{(d-i)^2}{d}
$$

$q_i$, ovvero la probabilit� che la popolazione $X_n$ venga ridotta di una unit� risulta essere l'evento:

$$
q_i = P(X_{n+1} = i - 1 | X_n = i) = P{NB} = \frac{(d-i)^2}{d}
$$

La probabilit� che la popolazione risulti stazionaria, $r_i$ risulta essere l'unione degli eventi $P(NN)$ e $P(BB)$ di stessa probabilit�:

$$
r_i = P(X_{n+1} = i | X_n = i) = P(NN) + P(BB) = 2 (\frac{i}{d})(\frac{d-i}{d}) 
$$

Verifichiamo quindi che si tratti veramente di una catena di markov, andando a vedere se la generica riga $i$ sommi a 1:

$$
p_i + r_i + q_i = \frac{(d-i)^2}{d} + 2 (\frac{i}{d})(\frac{d-i}{d} + \frac{(d-i)^2}{d} = 
$$

$$
= \frac{d^2+i^2-2id+i^2+2id-2i^2}{d^2} = \frac{d^2}{d^2} = 1
$$

Inizialmente $d$ biglie scelte a caso tra quelle nere e bianche sono poste nella prima urna e le restanti $d$ nella seconda. Notiamo anche che nella prima urna sono presenti solo biglie nere allora necessariamente si verificher� l'evento (NB), ovvero verr� estratta una biglia nera dalla prima urna e una bianca dalla seconda. Analogamente se $i = 0$, ovvero nella prima urna sono presenti esclusivamente biglie di colore bianco allora si deve necessariemente verificare l'evento (BN). Siamo quindi in presenta di una \textbf{catena di nascita e morte con barriere riflettenti}.

\section{Test}
Avendo ora stabilito che si tratta di una catena di nascita e morte con barriere riflettenti possiamo "istanziare" il problema per ogni d. Proviamo per d = 3. 

\end{document}

\documentclass{article}
\RequirePackage{amssymb}
\RequirePackage{amsmath}
\usepackage[latin1]{inputenc}

\title{Esercizio 1}
\author{Davide Angelocola}

\begin{document}
\maketitle

\section{Problema}
Dimostrare che la matrice:

$$A = \begin{pmatrix}2 & 1 & 1\cr 1 &-\frac{3}{2} & 1 &\cr 1 & 1 & -6  \end{pmatrix}$$

\begin{enumerate}
    \item � non singolare
    \item valutare $\mu_2(A)$ (minorazione e maggiorazione)
\end{enumerate}

\section{Svolgimento}

\subsection{1. $A$ non � singolare}
$A$ � una matrice simmetrica e reale, perci� possiamo affermare che
� normale ($A=A^H$). Ogni cerchio $C_w$ contiene almeno un autovalore. Dato che la matrice contiene 3 righe si hanno 3 cerchi, centrati in posizione $A_{i i}$:

\begin{enumerate}
    \item $C_1$ � centrato in 2 ($A_{1 1}$) e ha raggio $\sqrt{|1|^2+|1|^2} = \sqrt{2}$
    \item $C_2$ ha centro in $-\frac{3}{2}$ e ha sempre raggio $\sqrt{2}$
    \item infine $C_3$ � centrato in $-6$ e ha sempre raggio $\sqrt{2}$
\end{enumerate}

inoltre ciascuno di questi cerchi non "tocca" lo 0 poich� $C_1$ � compreso tra $[2-\sqrt{2}, 2+\sqrt{2}]$, $C_2$ fra $[\frac{-3}{2}-\sqrt{2},\frac{-3}{2}+\sqrt{2}]$ e $C_3$ tra $[-6-\sqrt{2},-6+\sqrt{2}]$. Poiche' $\sqrt{2}$ � minore di ogni centro si ha che $C_1$ � completamente positivo mentre $C_2$ e $C_3$ risultano negativi. Dato che $det(A) = \lambda_1 * \lambda_2 * \lambda_3$ e dato che nessun autovalore pu� essere zero si � dimostrato che la matrice $A$ � non singolare.

\subsection{2. Stima di $\mu_2(A)$}
Per determinare una maggiorazione e una minorazione del numero di condizionamento in norma 2, ovvero $\mu_2(A)$, si nota innanzitutto che $A$ � matrice normale. Poich� $A$ � una matrice reale e simmetrica risulta che:

$$A^HA=AA^H$$

e quindi $A$ � diagonalizzabile da una matrice $T$ unitaria. Sotto l'ipotesi di $A$ normale, il problema della ricerca degli autovalori di una matrice risulta ben condizionato.

Possiamo quindi scrivere la formula per 
$$\mu_2(A) = \frac{max|\lambda_i|}{min|\lambda_i|}$$

per $A$ si riduce a:

$$\mu_2(A) = \frac{|\lambda_1|}{|\lambda_2|}$$

per via del modulo si ottiene:

$$\frac{|-6-\sqrt{2}|}{|-\frac{3}{2}-\sqrt{2}|} \le \mu_2(A) \le \frac{|-6-\sqrt{2}|}{|-\frac{3}{2}+\sqrt{2}|}$$ 
\end{document}

\documentclass{article}
\RequirePackage{amssymb}
\RequirePackage{amsmath}
\usepackage[latin1]{inputenc}

\title{Esercizio 1}
\author{Davide Angelocola}

\begin{document}
\maketitle

\section{Problema}
Dimostrare che la matrice:

$$A = \begin{pmatrix}2 & 1 & 1\cr 1 &-\frac{3}{2} & 1 &\cr 1 & 1 & -6  \end{pmatrix}$$

\begin{enumerate}
    \item � non singolare
    \item valutare $\mu_2(A)$ (minorazione e maggiorazione)
\end{enumerate}

\section{Svolgimento}

\subsection{1. $A$ non � singolare}

Per poter dimostrare che $A$ sia non-singolare occorre studiare come sono posizionati gli autovalori. A tal fine si ricorre ai teoremi di \textit{Gershgorin}. Per il primo teorema di G. si ottengono due cerchi $C_1$ di centro $2$ e $C_2$ di centro $-\frac{3}{2}$ che contengono lo 0. Sempre tramite il primo teorema di G. si pu� affermare con certezza che $C_3$ non � un autovalore nulla in quanto � centrato in $-6$ e ha raggio $2$. 

Per il secondo teorema di G. si evince che in $C_1 \bigcup C_2$ vi sono 2 autovalori. Infine, per il terzo teorema di G., si dimostra che $C_1$ non contiene l'autovalore 0, poich� la matrice $A$ � irriducibile (non contiene zeri!). Rimane quindi solo $C_3$ come unico candidato a contenere un autovalore 0 che renderebbe la matrice $A$ singolare.

$A$ � una matrice simmetrica e reale, perci� possiamo affermare che
� normale, ovvero $A=A^H$. Poich� $A$ � normale si pu� applicare il teorema dei cerchi tilde (ha un nome questo teorema?) e poich� tramite i teoremi di G. abbiamo fatto vedere che $C_1$ e $C_3$ non possono contenere un autovalore 0, allora non rimane che verificare $\tilde{C_3}$. Esso sempre in $-\frac{3}{2}$ ma raggio $\sqrt{1^2+1^2}=\sqrt(2) < 3/2$, quindi non tocca lo $0$.

Infine poich�:

  $$det(A) = \lambda_1 * \lambda_2 * \lambda_3$$ 

e nessun autovalore pu� essere zero si � dimostrato che la matrice $A$ \textbf{� non-singolare}.

\subsection{2. Stima di $\mu_2(A)$}
Per determinare una maggiorazione e una minorazione del numero di condizionamento in norma 2, ovvero $\mu_2(A)$, si nota innanzitutto che $A$ � matrice normale. Poich� $A$ � una matrice reale e simmetrica risulta che:

$$A^HA=AA^H$$

e quindi $A$ � diagonalizzabile da una matrice $T$ unitaria. Sotto l'ipotesi di $A$ normale, il problema della ricerca degli autovalori di una matrice risulta ben condizionato.

Possiamo quindi scrivere la formula per 
$$\mu_2(A) = \frac{max|\lambda_i|}{min|\lambda_i|}$$

per $A$ si riduce a:

$$\mu_2(A) = \frac{|\lambda_1|}{|\lambda_2|}$$

per via del modulo si ottiene:

$$\frac{|-6-\sqrt{2}|}{|-\frac{3}{2}-\sqrt{2}|} \le \mu_2(A) \le \frac{|-6-\sqrt{2}|}{|-\frac{3}{2}+\sqrt{2}|}$$ 
\end{document}

\documentclass{article}
\RequirePackage{amssymb}
\RequirePackage{amsmath}
\usepackage[latin1]{inputenc}

\title{Appunti semplificati \\ per le Catene di Markov}
\author{Davide Angelocola}

\begin{document}
\maketitle

\section{Persistenza + Irriducibilit�}

Se $S$ � finita allora esiste uno stato persistente. Se $P$ � irriducibile tutti gli stati sono persistenti. Tempo medio di ritorno � finito ed �:

$$
v_i = \frac{1}{E_i[t_i]}
$$

\section{Misure invarianti}

Un vettore probabilistico $v$ con $v_i\ \in [0,1]$ e con $i\ \in S$ si dice invariante quando, assegnato $\pi_o = v$, si ha:

$$
\pi_n = v,\ \ \forall n \ge 1
$$ 

Ogni possibile distribuzione iniziale invariante assegna $0$ agli stati transienti.

\section{Teorema di Markov/Kakutani}

Una matrice di transizione su spazio finito ammette sempre almeno una misura invariante.

\section{Ergodicit�}

Se una catena � ergotica allora ammette un'unica misura invariante chiamata, misura stazionaria. Osserva che se trovo pi� misure invariante, ad esempio se ci sono pi� di due stati assorbenti, allora la dinamica non � ergotica. 

\section{Regolarit�}

Se $S$ � finita e $\exists\ n \ge 1 | p_{ij}^{(n)} > 0\ \forall i, j$ allora si dice regolare. 

Condizione necessaria e sufficiente per la regolarit�: $P$ irriducibile ed esista un loop.

\section{Regolarit� + Ergodicit�}

Se una catena � regolare allora ammette una misura stazionaria, ovvero � ergodica.

\section{Periodicit�}

Se una catena irriducibile allora:

$$
t_i = t
$$ 

ovvero il periodo � lo stesso per tutti gli stati.

\section{Periodicit� + Ergodicit�}

Condizione necessaria e sufficiente per ergodicit� di $P$ irriducibile:

$$
t = 1
$$ 

\section{Dinamiche bistocastiche}

Ogni dinamica bistocastica ammette la misura invariante uniforme.

\end{document}

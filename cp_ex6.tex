\documentclass{article}
\RequirePackage{amssymb}
\RequirePackage{amsmath}
\usepackage[latin1]{inputenc}

\title{Esercizio 6 di riepilogo}
\author{Davide Angelocola}

\begin{document}
\maketitle

\section{Testo del problema}

\'{E} assegnata la catena $(X_n)_{n\ge 0}$ con matrice di transizione:

$$
P = \begin{pmatrix}
0 & \frac{1}{2} & 0 & \frac{1}{2} \cr
\frac{1}{2} & 0 & \frac{1}{2} & 0 \cr
0 & 1 & 0 & 0 \cr
0 & 0 & 0 & 0 \cr
\end{pmatrix}
$$

e misura iniziale $\pi_0(1) = 1$.

\begin{itemize}
  \item i)  Studiare il comportamento della legge di $X_n$ per $n$ grande.
  \item i)) Calcolare il modo approssimato: $P(X_82 = 3, X_81 = 4)$ e $P(X_60 = 1, X_58 = 3)$. 
\end{itemize}

\section{i)}

Lo spazio degli stati � finito e irriducibile ma nessuno stato ha un loop. Si deduce quindi che la catena non � ergotica. Si nota anche che il periodo della catena � 2.

Ci sono quindi due misure invarianti, una per gli stati pari e una per gli stati dispari. 

\section{ii)}

\end{document}

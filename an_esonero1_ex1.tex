\documentclass{article}
\RequirePackage{amssymb}
\RequirePackage{amsmath}
\usepackage[latin1]{inputenc}

\title{Esercizio 1 - Esonero 1}
\author{Davide Angelocola}

\begin{document}
\maketitle

\section{Problema}

Sapendo che $B_4(t) = t^2(t-1)^2 - \frac{1}{30}$ calcolare $B_5(t)$ e trovare una formula compatta per $\sum_{x=1}^{n}{x^4}$.

\section{Svolgimento}

Al fine di determinare una formula chiusa per calcolare $\sum_{x=1}^{n}{x^4}$ si
deve determinare $B_5(x)$ in modo tale da scrivere la sommatoria come:

$$\sum_{x=1}^{n}{x^k} = \frac{1}{k+1} (B_{k+1}(n+1) - B_{k+1}(1))=\frac{1}{5} (B_5(n+1) - B_5(1)) $$\label{1}

Si vuole quindi calcolare un polinomio di Bernoulli tale che:

$$B(x+1) - B(x) = 5x^4$$

Per definizione $B_5(x) = 5\int_0^x{B_4(t)dt}$. Svolgendo i calcoli si ottiene:

$$B_5(x) = 5\int_0^x{t^2(t-1)^2-\frac{1}{30}dt}=$$
$$=5\int_0^x{t^4-2t^3+t^2-\frac{1}{30}dt}=$$

scrivendo la primitiva del polinomio in $t$ si ottiene:

$$=5[\frac{t^5}{5}-2\frac{t^4}{4}+\frac{t^3}{3}-\frac{1}{30}t]=$$
$$=5t^5-\frac{5}{2}t^4+\frac{5}{3}t^3-\frac{1}{6}t.$$

$\Box$

\section{Svolgimento (sbagliato)}

Al fine di determinare una formula chiusa per calcolare $\sum_{x=1}^{n}{x^4}$ si
deve determinare $B_5(x)$ in modo tale da scrivere la sommatoria come:

$$\sum_{x=1}^{n}{x^k} = \frac{1}{k+1} (B_{k+1}(n+1) - B_{k+1}(1))=\frac{1}{5} (B_5(n+1) - B_5(1)) $$\label{1}

Si vuole quindi calcolare un polinomio di Bernoulli tale che:

$$B(x+1) - B(x) = 5x^4$$

Si prende come polinomio B il polinomio generico, ottenendo:

$$a(x+1)^5+b(x+1)^4+c(x+1)^3+d(x+1)^2+e(x+1)+f - (ax^5+bx^4+cx^3+dx^2+ex+f) = 5x^4$$

Svolgendo i calcoli si ottiene:

$$5ax^4 + x^3(10a+4b) + x^2(10a+6b+3c)+x(5a+4b+3c+2d)+(a+b+c+d+e+f)=5x^4$$ 

isolando il termine di quarto grado si ottiene $a$:

$$5ax^4=5x^4 \Rightarrow a = 1$$

analogamente ponendo a 0 il coefficiente di $x^3$ si ricava $b$:

$$10a + 4b = 0$$ 
$$4b = -10$$ 
$$b = -\frac{5}{2}$$

similmente si ottengono i coefficienti di $c = -\frac{5}{3}, d = 0, e = \frac{19}{6}$. Si ottiene quindi il seguente polinomio:

$$p(x) = x^5 - -\frac{5}{2}x^4 + -\frac{5}{3}x^3 - -\frac{19}{6}x$$

Integrando nell'invertallo $[0, 1]$ si ottiene:

$$f = \int_0^1{x^5 -\frac{5}{2}x^4 -\frac{5}{3}x^3 -\frac{19}{6}x} = (\frac{x^6}{6} - \frac{5}{2}\frac{x^5}{5} -\frac{5}{3}\frac{x^4}{4} -  \frac{19}{6}\frac{x^2}{2})|_0^1 = \frac{3}{2}$$ 

Si ottiene dunque: 

$$B_5(X) = x^5 - \frac{5}{2}x^4 + \frac{5}{3}x^3 - \frac{x}{6}$$

Sostituendo quindi $B_5$ nella \ref{1} si ottiene quindi una formula chiusa per il calcolo della sommatoria voluta: 

$$\sum_{x=1}^{n}{x^4} = \frac{1}{5}(B_5(n+1) - B_5(1))$$

calcolando quindi $B_5$ nei punti richiesti: 

$$B_5(n+1) = (n+1)^5 - \frac{5}{2}(n+1)^4 + \frac{5}{3}(n+1)^3 - \frac{n+1}{6} =  ... = \frac{6n^5+15n^4+10n^3-n}{6}$$

$$B_5(1) = 1 - \frac{5}{2} + \frac{5}{3} - \frac{1}{6} = \frac{18}{3} = 3$$

infine:

$$\sum_{x=1}^{n}{x^4} = \frac{1}{5}(\frac{6n^5+15n^4+10n^3-n}{6}-3)=\frac{6n^5+15n^4+10n^3-n-18}{30}$$

\end{document}

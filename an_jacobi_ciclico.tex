\documentclass{article}
\RequirePackage{amssymb}
\RequirePackage{amsmath}
\usepackage[latin1]{inputenc}

\title{Metodo ciclico di Jacobi}
\author{Davide Angelocola}

\begin{document}
\maketitle

\section{Introduzione}



\section{Motivazione}

Il metodo di Jacobi nella sua forma originale richiede che si
analizzano $\frac{n (n - 1)}{2}$ per selezionare quello di modulo
maggiore. Per matrici grandi questo \`{e} un processo relativamente
lento, specialmente se si tratta di computer digitali.

\`{E} molto pi\`{u} conveniente selezionare le coppie $(i, j)$ in
qualche ordine ciclico. In questa sede si analizzano due ordini:

\begin{itemize}
\item (I) per righe
\item (II) per colonne
\end{itemize}



\section{Criterio per colonne}
\section{Criterio per righe}
\section{Conclusioni}

\begin{thebibliography}{99}
\bibitem{ForHen} G. E. Forsynthe, P. Henrici. \textit{The Cyclic
    Jacobi Method for Computing the Principal values of a complex
    matrix}
\end{thebibliography}

\end{document}

\documentclass{article}
\RequirePackage{amssymb}
\RequirePackage{amsmath}
\usepackage[latin1]{inputenc}

\title{Esercizio 3 di riepilogo}
\author{Davide Angelocola}

\begin{document}
\maketitle

\section{Testo del problema}

Calcola la probabilit� di raggiungere in un tempo finito lo stato 3 partendo dallo stato 1, per la catena con spazio degli starti $S = \{ 1, 2, 3, 4\}$ e con matrice di transizione:

$$
P = \begin{pmatrix}
0 & \frac{1}{2} & \frac{1}{2} & 0 \cr 
\frac{1}{2} & 0 & \frac{1}{4} & \frac{1}{4} \cr 
0 & 0 & 1 & 0 \cr 
0 & 0 & 0 & 1 \cr 
\end{pmatrix}
$$

E se la densit� iniziale fosse quella uniforme?

\section{Svolgimento}

Impostando il sistema di equazioni con $T={2,3}$ ottieniamo:

$$x_1 = \frac{1}{2} + \frac{1}{2}x_2$$
$$x_2 = \frac{1}{4} + \frac{1}{2}x_1$$

sostituendo $x_2$ nella prima equazione otteniamo:

$$x_1 = \frac{5}{6}$$

da cui:

$$x_2 = \frac{2}{3}$$

Quindi, partendo dallo stato 1 si viene assorbiti in tempo finito in 3 con probabilit� pari a:

$$x_1 = \lambda_1^{\{3\}} = \frac{5}{6}$$


Viceversa, con densit� iniziale uniforme, si dovrebbe fare la "media" tra tutti i possibili valori iniziali. Si nota subito per� che partire dallo proprio dallo stato 3 si raggiunge subito con probabilit� 1 mentre partendo dallo stato 4 non si raggiunge mai lo stato 3, essendo 4 assorbente.

Si ottiene dunque:

$$P(raggiungere 3 in tempo finito) = \sum_i{\pi_0(i) \lambda_i^{\{3\}}}$$

essendo la probabilit� uniforme si ha:

$$\frac{1}{4} (\pi_0(1) \lambda_1^{\{3\}} + \pi_0(1) \lambda_1^{\{3\}} + \pi_0(1) \lambda_1^{\{3\}} + \pi_0(1) \lambda_1^{\{3\}}) = $$

$$=\frac{1}{4} (\frac{5}{6} + \frac{2}{3} + 1 + 0) =$$

$$=\frac{5}{8}$$

\end{document}

\documentclass{article}
\RequirePackage{amssymb}
\RequirePackage{amsmath}
\usepackage[latin1]{inputenc}

\title{Esercizio 7}
\author{Davide Angelocola}

\begin{document}
\maketitle

\section{Problema}

\begin{description}
    \item [a] calcolare la probabilit� di assorbimento dello stato $0$ per la catena con matrice di transizione:

$$
P = 
\begin{pmatrix}
1 & 0 & 0 & 0                               \cr 
\frac{1}{3} & 0 & \frac{1}{3} & \frac{1}{3} \cr 
0 & \frac{1}{4} & \frac{1}{4} & \frac{1}{2} \cr 
0 & 0 & 0 & 1                               \cr 
\end{pmatrix}
$$

    \item [b] calcolare la probabilit� che la catena sia assorbita nello stato 0 se la densit� iniziale � quella uniforme

    \item [c] calcolare i tempi medi di assorbimento in $\{1, 3\}$
\end{description}


\subsection{\textbf{a}}

Al fine di determinare la probabilit� di assorbimento dello stato $0$ si usa la seguente formula:

\begin{equation} 
x_i = \sum_{j \in S \backslash T}{p_{i j}} + \sum_{j \in T}{p_{i j}x_j}, \forall i \in T
 \label{eq:1}
\end{equation}

Nel caso in esame l'insieme degli stati transienti risulta essere:

$$
T = \{ 1, 2 \}
$$

si ottiene quindi un sistema composto da due equazioni:

$$
\begin{cases}
1 & x_1 = \frac{1}{3} + \frac{1}{3}x_2        \\
2 & x_2 = 0 + \frac{1}{4}x_1 + \frac{1}{4}x_2 \\
\end{cases}
$$

Dalla seconda equazione si ricava che: 

$$x_2 = \frac{4}{3}\frac{1}{4}x_1 = \frac{1}{3}x_1$$ 

sostituendo $x_2$ nella prima equazione si ha:
 
$$x_1 - \frac{1}{9}x_1 = \frac{1}{3}$$. Infine, applicando la definizione di $\lambda$ si conclude che la probabilit� di essere assorbiti nello stato $0$ �:

$$\lambda^{\{0\}}_1 = \frac{3}{8} = x_1$$
$$\lambda^{\{0\}}_2 = \frac{1}{8} = x_2$$
 

\section{\textbf{b}}

Applicando la seguente equazione:

$$
P(T_0 < +\inf) = \sum_{j \in S}{\lambda_{j}^{\{0\}}p}  
$$

dove $p$ rappresenta la probabilit� della uniforme. 

Nel nostro caso si ottiene:

$$
P(T_0 < +\inf) = \lambda_0^{\{0\}} \frac{1}{4} + \lambda_1^{\{0\}} \frac{1}{4} + \lambda_2^{\{0\}} \frac{1}{4} + \lambda_3^{\{0\}} \frac{1}{4}  
$$

Poich� $\lambda_3^{\{0\}}$ � $0$ per definizione dato che lo stato 3 � assorbente. Si ottiene quindi:

$$ = \frac{1}{4} + \frac{3}{32} + \frac{1}{32} = $$
$$ = \frac{1}{4} + \frac{4}{32} = $$
$$ = \frac{3}{8} $$ 

\section{\textbf{c}}

Per calcolare i tempi medi di assorbimento in $\{0, 3\}$ si deve costruire un sistema di equazioni applicando:

$$
x_i = 1 + \sum_{h \in T}{p_{ih}x_h}
$$

per lo stato transiente 1 si ottiene:

$$
x_1 = 1 + \frac{1}{3}x_2
$$

mentre per lo stato transiete 2:

$$
x_2 = 1 + \frac{1}{4}x_1 + \frac{1}{4}x_2
$$

Sostituendo $x_1$ nella seconda si ottiene:

$$
x_2 = 1 + \frac{1}{4}(1 + \frac{1}{3}x_2) + \frac{1}{4} =
$$

$$
x_2 = \frac{5}{4}\frac{3}{2} = \frac{15}{8}
$$

sostituendo $\frac{15}{8}$ in $x_1$ si ottiene:

$$ 
x_1 = 1 + \frac{1}{3} \frac{15}{8} = 1 + \frac{5}{8} = \frac{13}{8} 
$$

In conclusione si ha che $E_1[T_C] = x_1 = \frac{13}{8}$ e $E_2[T_C] = x_2 = \frac{15}{8}$.

\end{document}

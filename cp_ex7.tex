\documentclass{article}
\RequirePackage{amssymb}
\RequirePackage{amsmath}
\usepackage[latin1]{inputenc}

\title{Esercizio 7}
\author{Davide Angelocola}

\begin{document}
\maketitle

\section{Problema}

\begin{description}
    \item [a] calcolare la probabilit� di assorbimento dello stato $0$ per la catena con matrice di transizione:

$$
P = 
\begin{pmatrix}
1 & 0 & 0 & 0                               \cr 
\frac{1}{3} & 0 & \frac{1}{3} & \frac{1}{3} \cr 
0 & \frac{1}{4} & \frac{1}{4} & \frac{1}{2} \cr 
0 & 0 & 0 & 1                               \cr 
\end{pmatrix}
$$

    \item [b] calcolare la probabilit� che la catena sia assorbita nello stato 0 se la densit� iniziale � quella uniforme

    \item [c] calcolare i tempi medi di assorbimento in $\{1, 3\}$
\end{description}

\subsection{Svolgimento punto \textbf{a}}

Al fine di determinare la probabilit� di assorbimento dello stato $0$ si usa la seguente formula:

$$
x_i = \sum_{j \in S \backslash T}{p_{i j}} + \sum_{j \in T}{p_{i j}x_j}, \forall i \in T
$$

Nel caso in esame l'insieme degli stati transienti risulta essere:

$$
T = \{ 1, 2 \}
$$

si ottiene quindi un sistema composto da due equazioni:

$$
\begin{cases}
1 & x_1 = \frac{1}{3} + \frac{1}{3}x_2        \\
2 & x_2 = 0 + \frac{1}{4}x_1 + \frac{1}{4}x_2 \\
\end{cases}
$$

Dalla seconda equazione si ricava che: 

$$x_2 = \frac{4}{3}\frac{1}{4}x_1 = \frac{1}{3}x_1$$ 

sostituendo $x_2$ nella prima equazione si ha:
 
$$x_1 - \frac{1}{9}x_1 = \frac{1}{3}$$. Infine, applicando la definizione di $\lambda$ si conclude che la probabilit� di essere assorbiti nello stato $0$ �:

$$\lambda^{\{0\}}_1 = \frac{3}{8} = x_1$$
$$\lambda^{\{0\}}_2 = \frac{1}{8} = x_2$$
 
\end{document}

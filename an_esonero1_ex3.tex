\documentclass{article}
\RequirePackage{amssymb}
\RequirePackage{amsmath}
\usepackage[latin1]{inputenc}

\title{Esercizio 3 - Esonero 1}
\author{Davide Angelocola}

\begin{document}
\maketitle

\section{Problema}
Sia 

$$A = \begin{pmatrix}2 & \frac{2}{3} & 0\cr \frac{2}{3} & -1 & -\frac{2}{3}\cr 0 & -\frac{2}{3} & 3 \end{pmatrix}$$

\begin{enumerate}
    \item dimostrare che $A$ � non singolare e dare una limitazione per $\mu_2(A)$
    \item focalizzare l'autovalore di $A$ con modulo minimo
\end{enumerate}

\section{Svolgimento}

\subsection{1. $A$ non singolare}

Per il primo teorema di G. si considerano 3 cerchi, $C_1, C_2, C_3$:
\begin{itemize}
    \item $C_1$ ha centro in $2$ e raggio $\frac{2}{3}$
    \item $C_2$ ha centro in $-1$ e raggio $|\frac{2}{3}|+|-\frac{2}{3}|=\frac{4}{3}$
    \item $C_3$ ha centro in $3$ e raggio $\frac{2}{3}$
\end{itemize}

\'{E} evidente che il secondo cerchio, $C_2$, potrebbe includere lo zero. Inoltre,  per il secondo teorema di \textit{Gerschgorin}, in $C_1 \bigcup C_3$ ci sono due autovalori. Si pu� applicare anche il terzo teorema di G. poich� la matrice $A$ � irriducibile (anche se ha elementi nulli, $\forall i, j \in V \exists e \in E$ che unisce i a j) i punti di frontiera non possono essere autovalori di $A$. 

$A$ � simmetrica e reale, segue quindi che essa � normale. Si possono quindi considerare anche i cerchi $\tilde{C_1}, \tilde{C_2} e \tilde{C_3}$:

\begin{itemize}
    \item $\tilde{C_1}$ ha centro in $2$ e raggio $\sqrt{(\frac{2}{3})^2} = \frac{2}{3}$ 
    \item $\tilde{C_2}$ ha centro in $-1$ e raggio $\sqrt{(\frac{2}{3})^2+(-\frac{2}{3})^2}=\sqrt{\frac{4}{9}} = \frac{2}{3}$
    \item $\tilde{C_3}$ ha centro in $3$ e raggio $\sqrt{(\frac{2}{3})^2} = \frac{2}{3}$
\end{itemize}

\'{E} subito evidente che i cerchi $\tilde{C_1}$ e $\tilde{C_3}$ corrispondono, rispettivamente, con i cerchi $C_1$ e $C_3$. Questo secondo teorema bench� non ci assicuri che in $\tilde{C_1}$ ci sia un autovalore sappiamo, per il secondo teorema di G. che nell'unione $C_1 \cup C_2$ ve ne sono due. Poich� il cerchio $\tilde{C_1}$ non "tocca" lo zero e dato che per il secondo teorema di G. ve ne deve essere uno nel cerchio $C_1$, ma il cerchio $\tilde{C_1}$ ha modulo minore di 1 e quindi si � dimostrato che non esistono autovalori nulli, pertanto la matrice � non-singolare.

\subsection{1.1 maggiorazione e minorazione per $\mu_2$}

$A$ � normale, possiamo quindi scrivere la formula per: 

$$\mu_2(A) = \frac{max|\lambda_3|}{min|\lambda_1|}$$

al fine di determinare una maggiorazione si scelgono due valori $\lambda_i$ tali da massimizzare il rapporto:

$$\mu_2(A) \le \frac{3+\sqrt{\frac{2}{3}}}{-1-\sqrt{\frac{4}{3}}}$$

analogamente si calcola una minorazione di $\mu_2$ minimizzando il rapporto, ovvero prendendo il pi� grande autovalore al denominatore ed il pi� piccolo al numeratore. 

\section{Svolgimento al computer}

\begin{verbatim}
In[91]:= Eigenvalues[{{2,2/3,0},{2/3,-1,-2/3},{0,-2/3,3}} ]
Out[91]= {Root[74+#1-36 #1^2+9 #1^3&,3],Root[74+#1-36 #1^2+9    #1^3&,2],Root[74+#1-36 #1^2+9 #1^3&,1]}

In[92]:= N[%]
Out[92]= {3.1194,2.12247,-1.24187}
\end{verbatim}

\end{document}

\documentclass{article}
\RequirePackage{amssymb}
\RequirePackage{amsmath}
\usepackage[latin1]{inputenc}
\usepackage{url}
\usepackage{bbding} %% per usare \Cross
\usepackage{stmaryrd}

\title{Sul Metodo Iterativo di Jacobi per il calcolo di autovalori}
\author{Davide Angelocola}


\begin{document}
\maketitle
\begin{abstract}
Si presentano vari criteri di enumerazione del pivot da usare nel Metodo Iterativo di Jacobi per la diagonalizzazione di una matrice.
\end{abstract}

\section{Introduzione}

Il metodo di Jacobi per il calcolo degli autovalori di una matrice simmetrica � un algoritmo numerico iterativo. 

Sia $A$ una matrice $nxn$ simmetrica. Si definisce \cite{difiore} quindi la serie di iterazioni definita da:

$$
A_0 = A \longrightarrow A_1 = S_1^TA_0S_1 \longrightarrow ... \longrightarrow A_k = S_k^TA_{k-1}S_k.
$$ 

e si sceglie $S_k$ in modo tale che:

\begin{enumerate}
  \item $S_k^T = S_k^{-1}$ con $A_k$ simile e simmetrica a $A_{k-1}$
  \item $(A_k)_{p_k q_k} = 0$   
\end{enumerate} 

Il presente documento indaga sui criteri di scelta di $p_k$ e $q_k$ per i quali il metodo converge. Per ogni criterio di scelta si dimostra che:

$$
\{A_k\}_{k=0}^{+\infty}
$$

$$
(A_k)_{p_k q_k} \longrightarrow 0, p_k \ne q_k
$$

$$
(A_k)_{p_k q_k} \longrightarrow \lambda, p_k = q_k
$$

Nella sezione \ref{sec:othermethods} presenteremo dei metodi iterativi diversi da quello di Jacobi.

\section{Criterio modulo pi� grande}

Al generico passo si sceglie l'elemento pi� grande in modulo del passo precedente. La dimostrazione di convergenza del metodo di Jacobi con questo criterio di scelta del pivot � discussa in \cite{difiore}. Inoltre la convergenza del metodo � quadratica:

$$
da scrivere
$$

\section{Criterio sequenziale}

Al generico passo si scelgono $p_k$ e $q_k$ tali che:

$$
p_k < q_k
$$

La dimostrazione della convergenza di questo metodo � stata discussa da 
(citazione).

\section{Criterio Loizou}
%% da "On the Quadratic Convergence of the Jacobi Method for Normal Matrices"
%% http://comjnl.oxfordjournals.org/cgi/content/abstract/15/3/274
The pivot pair (p, q) is chosen so that the sum of the absolute squares of the elements in positions (p, q) and (q, p) is greatest.

\section{Criterio quasi-ciclico}
%% http://www.springerlink.com/content/gk0626782u774386

\section{Criterio JBD}

JDB sta per Joint-Block Diagonlization. Ad ogni iterazione si sceglie il pivot $(p, q)$ che assicura un decremento massimo di $C_{jbd}$. Questo richiede il calcolo di tutte le differenze $boff(A_k)$.
 
\section{Parallelizzazione}

\'{E} interessante notare che il metodo di Jacobi si presta molto bene ad essere usato in ambito parallelo.

\section{Altri metodi}\label{sec:othermethods}

Per completezza citiamo gli altri metodi iterativi:

\begin{itemize}
  \item medoto della potenza 
  \item metodo di Lanczos  
  \item metodo di Arnoldi
  \item metodo di Davidson
  \item metodo di Jacobi-Davidson\cite{jacdav}
\end{itemize}

\begin{thebibliography}{5}
  \bibitem{difiore} Dispense corso Analisi Numerica 3, Di Fiore, 2007 
 
  \bibitem{jacdav} A Jacobi-Davidson Iteration Method for Linear Eigenvalue problems, G. L. G. Sleijpen \Cross, H. A. Van der Vorst \Cross , 1996

%% altre fonti
%% \url{http://en.wikipedia.org/wiki/Jacobi_eigenvalue_algorithm}
%% \url{http://www.maths.lancs.ac.uk/~gilbert/m306c/node17.html}
%% \url{http://math.fullerton.edu/mathews/n2003/JacobiMethodMod.html}
%% \url{http://www.springerlink.com/content/v8p614u1201n778t}
  
\end{thebibliography}

\end{document}

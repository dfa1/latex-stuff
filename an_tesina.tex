%% TODO :
%%   - dimostrazione convergenza criterio ciclico (da cercare) 
%%   - metodo per matrici normali + fare esercizi a fine tesina
%%   - studiare la convergenza del metodo RTZ e del metodo Zhou
%%   - P.J. Eberlein "On the Schur decomposition of a matrixfor parallel 
%%     computation"
%%   - M. Mantharam and P.J. Eberlein "The real two-zero algorithm: a parallel
%%     algorithm to reduce a real matrix to a real Schur form"
%%   - G. W. Stewart "A Jacobi-like algorithm for computing the Schur 
%%     decomposition of a nonhermitian matrix"
\documentclass{article}
\RequirePackage{amssymb}
\RequirePackage{amsmath}
\usepackage[latin1]{inputenc}
\usepackage{url}
\usepackage{stmaryrd}

\title{Sul Metodo Iterativo di Jacobi \\per il Calcolo degli Autovalori}
\author{Davide Angelocola}

\begin{document}
\maketitle
\begin{abstract}
Presentiamo diversi criteri di enumerazione del pivot per il Metodo Iterativo di Jacobi per il calcolo degli autovalori. Vengono inoltre presentate le propriet� di convergenza e dell'errore mentre, dei criteri pi� significativi, viene esibita anche la dimostrazione. Concludiamo, con la sezione \ref{sec:othermethods}, presentando brevemente metodi per il calcolo degli autovalori \textbf{alternativi} a quello di Jacobi.
\end{abstract}

\section{Introduzione}

Il metodo di Jacobi per il calcolo degli autovalori di una matrice � un algoritmo numerico iterativo, ovvero che arriva alla soluzione per approssimazioni successive finch� l'errore � sotto una certa soglia.

Sia $A$ una matrice $n \times n$ simmetrica. Si definisce \cite{difiore} quindi la serie di iterazioni definita da:

$$
A_0 = A \longrightarrow A_1 = S_1^TA_0S_1 \longrightarrow ... \longrightarrow A_k = S_k^TA_{k-1}S_k.
$$ 

e $S_k$:

\begin{enumerate}
  \item $S_k^T = S_k^{-1}$ in modo tale che $A_k$ sia simile a $A_{k-1}$
  \item $(A_k)_{p_k q_k} = 0$   
\end{enumerate} 

Indaghiamo quindi sui criteri di scelta degli indici $p_k$ e $q_k$, per ogni passo k, per i quali si ottiene convergenza. Per ogni criterio di scelta si dimostra che:

$$
\{A_k\}_{k=0}^{+\infty}
$$

$$
(A_k)_{i,j} \longrightarrow 0, i \ne j
$$

$$
(A_k)_{i,j} \longrightarrow \lambda, i = j
$$

\section{Criterio modulo pi� grande}

Al generico passo $k$ si sceglie l'elemento pi� grande in modulo della matrice $A_{k-1}$ al passo precedente:

$$
|(A_{k-1})_{p_k, q_k}| = max_{i,j}|(A_{k-1})_{i, j}|
$$

La dimostrazione di convergenza del metodo di Jacobi con questo criterio di scelta del pivot � discussa in \cite{difiore}.

Inoltre la convergenza del metodo � quadratica:

$$
TODO
$$

\section{Criterio ciclico}

Al generico passo $k$ si scelgono due indici $p_k$ e $q_k$ tali che:

$$
p_k < q_k
$$

In questo modo si ottengono, in sequenza tutti gli elementi sopra la diagionale. Ad esempio, sia $A$ una matrice $n \times n$, vengono presi ciclicamente gli elementi:

$$
A_{1,2} A_{1,3} A_{1,4} A_{2,3} A_{2,4} A_{n-1,n} A_{1,2} A_{1,3} A_{1,4} ... 
$$ 

\'{E} evidente che la sequenza dei pivot per una matrice $n \times n$ sia lunga \textit{almeno} $\frac{n^2 - n}{2}$.

La dimostrazione della convergenza di questo metodo � stata discussa da 
(citazione).

\section{Criterio per matrici normali}

I metodi basati su Jacobi hanno attratto l'attenzione poich� hanno un altro grado di parallelismo e possono, sotto determinate condizioni, fornire valori pi� accurati del metodo QR. Discuteremo un metodo \cite{zhou} simile a quello di Jacobi  per la diagonalizzazione di matrici reali normali. Questo metodo usa sono operazioni nei reali e trasformazioni simili, risultando in una convergenza quadratica.

\section{Criterio Loizou}

Il pivot $A_{p_k q_k}$ � scelto in modo tale che la somma del quadrato degli elementi $A_{p_k q_k}$ e $A_{q_k p_k}$ sia, in modulo maggiore. La convergenza del metodo di Jacobi usando questo criterio di enumerazione dei pivot � descritta da \cite{loizou}.

\section{Criterio quasi-ciclico}

Questo critero � usato in un metodo "speciale" di Jacobi, chiamato dal suo autore V. Hari, quasi-ciclico \cite{hari}. 

\section{Criterio JBD}

JDB sta per Joint-Block Diagonlization. Ad ogni iterazione si sceglie il pivot $(p, q)$ che assicura un decremento massimo di $C_{jbd}$. Questo richiede il calcolo di tutte le differenze $boff(A_k)$.
 
\section{Criterio Hamiltoniano}

Recentemente (2008) � stato presentato un criterio di scelta per il metodo di Jacobi che sfrutta la definizione delle matrici Hamiltoniane \cite{mehl}.

Una matrice $H \in \mathbb{R}^{2n \times 2n}$ si dice Hamiltoniana se $H^TJ + JH = 0$ dove:

$$
J = \begin{pmatrix}
0 & I_n \cr 
-I_n & 0 \cr 
\end{pmatrix}
$$ 

o in modo equivalente:

$$
H = \begin{pmatrix}
A	 & C \cr 
D & -A^T \cr 
\end{pmatrix}
$$

dove $A$, $C$ e $D$ sono matrici $n x n$ e $C = C^T$ e $D = D^T$.
 
Infine presentiamo varie strategie di scelta dei pivot:

\begin{itemize}
  \item bottom-to-top
  \item top-to-bottom
  \item out-to-in 
\end{itemize}

\section{Parallelizzazione}

\'{E} interessante notare che il metodo di Jacobi si presta molto bene ad essere usato in ambito parallelo. 

\section{Metodi simili}\label{sec:othermethods}

Per completezza citiamo metodi iterativi "simili" a Jacobi:

\begin{itemize}
  \item calcolo della forma di Schur form per matrici complesse (Greenstadt, 1955, Eberlein, 1962/87, Stewart 1985)
  \item diagonalizzazione di matrici normali (Goldstein, Hurwitz, 1959)
  \item Hamiltonian matrices (Byers, 1986, Bunse-Gerstner Fabender, 1997)  
  \item matrici doppiamente strutturate (Fabender, Mackey, Mackey, 2001)
  \item ... molti altri ... Drmac, Veselic, ....
\end{itemize}

Tali metodi presentano tutti diverse propriet� di convergenza che non verranno trattati in questa sede.

\section{Metodi alternativi}
Si presentano infine metodi alternativi a quello di Jacobi ma che producono sempre una diagonalizzazione della matrice iniziale. Tali metodi sono:

\begin{itemize}
  \item medoto della potenza 
  \item metodo di Lanczos  
  \item metodo di Arnoldi
  \item metodo di Davidson
  \item metodo di Jacobi-Davidson\cite{jacdav}
\end{itemize}

\section{Esercizi}
\subsection{Lower bound moltiplicazione tra complessi}
dimostrare che la moltiplicazione tra due complessi usa almeno 3 moltiplicazioni reali

\subsection{Matrici skew-simmetriche}
dimostrare che la matrice 
$$\begin{pmatrix}
0 & 1 \cr 
-1 & 0 \cr 
\end{pmatrix}
$$
� skew simmetrica; $A^T = -A$

\subsection{Matrici quasi normali}
Trovare la definizione.

\subsection{Matrici a Blocchi}
Come ridurre una matrice $4 \times 4$ a una matrice diagonale a blocchi $2 \times 2$ usando una trasformazione simmetrica ortogonale? 

\subsection{Thm}

Sia a A $\in n \times n$ a coefficienti reali. Allora se $\lambda*$ � autovalore di A allora necessariamente anche $\signed{\lambda*}$ � autovalore di A. (dove $\signed{\lambda}$ indica il coniugato).

Da questo teorema deriva che ogni matrice $3 \times 3$ contiene almeno un autovalore reale.

\subsection{Definizione polinomio caratteristico}

$$
A = \begin{pmatrix}
0 & 1 \cr 
-1 & 0 \cr 
\end{pmatrix}
$$

gli autovalori sono le soluzioni del polinomio $\lambda^2 + 1$ che � ottenuto da:

$$
p(\lambda) = \lambda - AI =
\begin{pmatrix}
\lambda & -1 \cr 
1 & \lambda \cr 
\end{pmatrix}
$$

\begin{thebibliography}{5}

  \bibitem{difiore} C. Di Fiore, Dispense corso Analisi Numerica 3, 2007 
 
  \bibitem{jacdav} G. L. G. Sleijpen, H. A. Van der Vorst, A Jacobi-Davidson Iteration Method for Linear Eigenvalue problems, 1996

  \bibitem{zhou} Zhou, B. B. and Brent, R. P., An efficient method for computing eigenvalues of a real normal matrix, 2003, \url{http://portal.acm.org/citation.cfm?id=876703#}

  \bibitem{loizou} G. Loizou, On the Quadratic Convergence of the Jacobi Method for Normal Matrices, 1972, \url{http://comjnl.oxfordjournals.org/cgi/content/abstract/15/3/274}

  \bibitem{hari} V. Hari, Quadratic convergence of a special quasi-cyclic Jacobi method, 2005, \url{http://www.springerlink.com/content/gk0626782u774386}.
  
  \bibitem{mehl} C. Mehl, Hamiltonian Jacobi methods: sweeping away convergence problems, 2008, \url{http://www.math.tu-berlin.de/~mehl/talks/harburg.pdf} 

%% altre fonti
%% \url{http://en.wikipedia.org/wiki/Jacobi_eigenvalue_algorithm}
%% \url{http://www.maths.lancs.ac.uk/~gilbert/m306c/node17.html}
%% \url{http://math.fullerton.edu/mathews/n2003/JacobiMethodMod.html}
%% \url{http://www.springerlink.com/content/v8p614u1201n778t}
  
\end{thebibliography}

\end{document}

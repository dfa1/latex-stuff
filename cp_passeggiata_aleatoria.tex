\documentclass{article}
\RequirePackage{amssymb}
\RequirePackage{amsmath}

\title{Matrice di transizione in 2 passi per la passeggiata aleatoria su $\mathbb{Z}$}
\author{Davide Angelocola}

\begin{document}
\maketitle

\section{Nota bene}
La soluzione dell'esercizio \'{e} palesemente \textbf{sbagliata}, poich\'{e} si sono fatti \textsl{diversi} errori:
\begin{itemize}
    \item si \'{e} assunto $r=0$, si deve esplicitare per il caso $r\neq0$
    \item posso rimanere fermo con due eventi, andando prima a destra e poi tornando e, simmetrimente, andando a sinistra e tornando: essi sono due eventi indipendenti di probabilit\'{a} totale pari a $2pq$
    \item terzo e ultimo errore, si \'{e} fatto un conto su una matrice bi-infinata... come se fosse finita e ci\'{o} non \'{e} corretto
\end{itemize}

\section{Problema}
Scrivere la matrice di transizione in 2 passi per la passeggiata aleatoria su  $\mathbb{Z}$ quando $f(i)= \begin{cases}
	p & i = +1     \\
	q & i = -1     \\
	r & i = 0      \\
	0 & altrimenti 
	\end{cases}
$

\section{Svolgimento a mano}

Si considera una porzione arbitraria di $\mathbb{Z}$, ad esempio l'intervallo $[0, 3]$, ottenendo quindi una matrice $4x4$:

$$P = \begin{pmatrix}0 & p & 0 & 0\cr q & 0 & p & 0\cr 0 & q & 0 & p \cr 0 & 0 & q & 0 \end{pmatrix}$$

per 2 passi si ottiene:

$$P^2 = \begin{pmatrix}0 & p & 0 & 0\cr q & 0 & p & 0\cr 0 & q & 0 & p \cr 0 & 0 & q & 0 \end{pmatrix} * \begin{pmatrix}0 & p & 0 & 0\cr q & 0 & p & 0\cr 0 & q & 0 & p \cr 0 & 0 & q & 0 \end{pmatrix} = \begin{pmatrix}pq & 0 & p^2 & 0\cr 0 & pq & 0 & p^2\cr q^2 & 0 & pq & 0 \cr 0 & q^2 & 0 & pq \end{pmatrix} $$

mentre per 3 passi si ottiene:

$$P^3 = P^2 * P =  \begin{pmatrix}pq & 0 & p^2 & 0\cr 0 & pq & 0 & p^2\cr q^2 & 0 & pq & 0 \cr 0 & q^2 & 0 & pq \end{pmatrix} * \begin{pmatrix}pq & 0 & p^2 & 0\cr 0 & pq & 0 & p^2\cr q^2 & 0 & pq & 0 \cr 0 & q^2 & 0 & pq \end{pmatrix} =  \begin{pmatrix}0 & p^2q & 0 & p^3\cr pq^2 & 0 & p^2q & 0\cr 0 & 2pq^2 & 0 & p^2q \cr q^3 & 0 & 2pq^2 & 0\end{pmatrix}$$

La generica riga dispari di questa matrice risulta pertanto essere:

$$ ... q^3 0 pq^2 0 p^2q 0 p^3 ... $$ 

con l'elemento $0 di \mathbb{Z}$ 0 di probabilit\'{a} 0, poiche' non e' possibile raggiungerlo con un numero dispari di passi. 

\section{Svolgimento con Maxima}
Si considera una porzione di $\mathbb{Z}$, ad esempio l'intervallo $[-3, +3]$, ottenendo quindi una matrice $7x7$:

$$P = \begin{pmatrix}0 & p & 0 & 0 & 0 & 0 & 0\cr q & 0 & p & 0 & 0 & 0 & 0\cr 0 & q & 0 & p & 0 & 0 & 0\cr 0 & 0 & q & 0 & p & 0 & 0\cr 0 & 0 & 0 & q & 0 & p & 0\cr 0 & 0 & 0 & 0 & q & 0 & p\cr 0 & 0 & 0 & 0 & 0 & q & 0\end{pmatrix}$$

in 2 passi si ottiene:

$$P^2 = \begin{pmatrix}0 & {p}^{2} & 0 & 0 & 0 & 0 & 0\cr {q}^{2} & 0 & {p}^{2} & 0 & 0 & 0 & 0\cr 0 & {q}^{2} & 0 & {p}^{2} & 0 & 0 & 0\cr 0 & 0 & {q}^{2} & 0 & {p}^{2} & 0 & 0\cr 0 & 0 & 0 & {q}^{2} & 0 & {p}^{2} & 0\cr 0 & 0 & 0 & 0 & {q}^{2} & 0 & {p}^{2}\cr 0 & 0 & 0 & 0 & 0 & {q}^{2} & 0\end{pmatrix}$$

mentre in 3 passi:

$$P^3 = \begin{pmatrix}0 & {p}^{3} & 0 & 0 & 0 & 0 & 0\cr {q}^{3} & 0 & {p}^{3} & 0 & 0 & 0 & 0\cr 0 & {q}^{3} & 0 & {p}^{3} & 0 & 0 & 0\cr 0 & 0 & {q}^{3} & 0 & {p}^{3} & 0 & 0\cr 0 & 0 & 0 & {q}^{3} & 0 & {p}^{3} & 0\cr 0 & 0 & 0 & 0 & {q}^{3} & 0 & {p}^{3}\cr 0 & 0 & 0 & 0 & 0 & {q}^{3} & 0\end{pmatrix}$$

risulta quindi evidente che, in generale, per $n$ passi si ottiene la seguente matrice di transizione:

$$P^n=\begin{pmatrix}0 & {p}^{n} & 0 & 0 & 0 & 0 & 0\cr {q}^{n} & 0 & {p}^{n} & 0 & 0 & 0 & 0\cr 0 & {q}^{n} & 0 & {p}^{n} & 0 & 0 & 0\cr 0 & 0 & {q}^{n} & 0 & {p}^{n} & 0 & 0\cr 0 & 0 & 0 & {q}^{n} & 0 & {p}^{n} & 0\cr 0 & 0 & 0 & 0 & {q}^{n} & 0 & {p}^{n}\cr 0 & 0 & 0 & 0 & 0 & {q}^{n} & 0\end{pmatrix}$$ 

\section{Considerazioni}

La matrice di transizione determinata con Maxima secondo me e' sbagliata, perche' alla riga 1 non ha una probabilita' a distanza 3, ma solamente a distanza 1. Ma essa rappresenta una matrice stocastica poiche' per ogni $n$ si verifica che: 

$$ p^n + q^n = 1^n $$

La matrice di transizione determinata a mano invece \'{e} intuitivamente corretta, poich\'{e} per saltare, ad esempio da 0 e 2, occorre saltare due volte a destra, il che avviene con probabilit\'{a} $p*p$. Inoltre essa considera l'evento "rimanere nella posizione corrente" che si verifica andando una volta avanti e una indietro. Questi eventi si verificano, rispettivamente, con probabilit\'{a} $p$ e $q$ e ci\'{o} giustifica gli elementi sulla diagonale pari a $pq$.

\end{document}
